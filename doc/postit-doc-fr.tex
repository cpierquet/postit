% !TeX TXS-program:compile = txs:///arara
% arara: pdflatex: {shell: yes, synctex: no, interaction: batchmode}
% arara: pdflatex: {shell: yes, synctex: no, interaction: batchmode} if found('log', '(undefined references|Please rerun|Rerun to get)')

\documentclass[french,a4paper,11pt]{article}
\usepackage[margin=2cm,includefoot]{geometry}
\def\TPversion{0.1.3}
\def\TPdate{12 juin 2023}
\usepackage[utf8]{inputenc}
\usepackage[T1]{fontenc}
\usepackage{amsmath,amssymb}
\usepackage{postit}
\usepackage{awesomebox}
\usepackage{fontawesome5}
\usepackage{footnote}
\makesavenoteenv{tabular}
\usepackage{enumitem}
\usepackage{tabularray}
\usepackage{wrapstuff}
\usepackage{lipsum}
\usepackage{fancyvrb}
\usepackage{fancyhdr}
\fancyhf{}
\renewcommand{\headrulewidth}{0pt}
\lfoot{\sffamily\small [postit]}
\cfoot{\sffamily\small - \thepage{} -}
\rfoot{\hyperlink{matoc}{\small\faArrowAltCircleUp[regular]}}

%\usepackage{hvlogos}
\usepackage{hologo}
\providecommand\tikzlogo{Ti\textit{k}Z}
\providecommand\TeXLive{\TeX{}Live\xspace}
\providecommand\PSTricks{\textsf{PSTricks}\xspace}
\let\pstricks\PSTricks
\let\TikZ\tikzlogo
\newcommand\TableauDocumentation{%
	\begin{tblr}{width=\linewidth,colspec={X[c]X[c]X[c]X[c]X[c]X[c]},cells={font=\sffamily}}
		{\LARGE \LaTeX} & & & & &\\
		& {\LARGE \hologo{pdfLaTeX}} & & & & \\
		& & {\LARGE \hologo{LuaLaTeX}} & & & \\
		& & & {\LARGE \TikZ} & & \\
		& & & & {\LARGE \TeXLive} & \\
		& & & & & {\LARGE \hologo{MiKTeX}} \\
	\end{tblr}
}

\usepackage{hyperref}
\urlstyle{same}
\hypersetup{pdfborder=0 0 0}
\setlength{\parindent}{0pt}
\definecolor{LightGray}{gray}{0.9}

\usepackage{babel}
\AddThinSpaceBeforeFootnotes
\FrenchFootnotes

\usepackage{listings}

\usepackage{newverbs}
\newverbcommand{\motcletex}{\color{cyan!75!black}}{}
\newverbcommand{\packagetex}{\color{violet!75!black}}{}

\tcbuselibrary{listingsutf8}
\newtcblisting{DemoCode}[1][]{%
	enhanced,width=0.95\linewidth,center,%
	bicolor,size=title,%
	colback=cyan!2!white,%
	colbacklower=cyan!1!white,%
	colframe=cyan!75!black,%
	listing options={%
		breaklines=true,%
		breakatwhitespace=true,%
		style=tcblatex,basicstyle=\small\ttfamily,%
		tabsize=4,%
		commentstyle={\itshape\color{gray}},
		keywordstyle={\color{blue}},%
		classoffset=0,%
		keywords={},%
		alsoletter={-},%
		keywordstyle={\color{blue}},%
		classoffset=1,%
		alsoletter={-},%
		morekeywords={center,justify,\lipsum},%
		keywordstyle={\color{violet}},%
		classoffset=2,%
		alsoletter={-},%
		morekeywords={PostIt,\MiniPostIt},%
		keywordstyle={\color{green!50!black}},%
		classoffset=3,%
		morekeywords={Couleur,CouleurAttache,Attache,Largeur,Hauteur,Inclinaison,Ombre,Coin,DecalAttache,AlignementH,AlignementV,AlignementPostIt,Bordure,ExtraMargeDroite,Rendu,Titre,PoliceTitre,RappelPostIt},%
		keywordstyle={\color{orange}}
	},%
	#1
}

\tcbset{vignettes/.style={%
	nobeforeafter,box align=base,boxsep=0pt,enhanced,sharp corners=all,rounded corners=southeast,%
	boxrule=0.75pt,left=7pt,right=1pt,top=0pt,bottom=0.25pt,%
	}
}

\tcbset{vignetteMaJ/.style={%
	fontupper={\vphantom{pf}\footnotesize\ttfamily},
	vignettes,colframe=purple!50!black,coltitle=white,colback=purple!10,%
	overlay={\begin{tcbclipinterior}%
			\fill[fill=purple!75]($(interior.south west)$) rectangle node[rotate=90]{\tiny \sffamily{\textcolor{black}{\scalebox{0.66}[0.66]{\textbf{MàJ}}}}} ($(interior.north west)+(5pt,0pt)$);%
	\end{tcbclipinterior}}
	}
}

\newcommand\Cle[1]{{\small\sffamily\textlangle \textcolor{orange}{#1}\textrangle}}
\newcommand\cmaj[1]{\tcbox[vignetteMaJ]{#1}\xspace}

\begin{document}

\setlength{\aweboxleftmargin}{0.07\linewidth}
\setlength{\aweboxcontentwidth}{0.93\linewidth}
\setlength{\aweboxvskip}{8pt}

\pagestyle{fancy}

\thispagestyle{empty}

\vspace{2cm}

\begin{center}
	\begin{minipage}{0.75\linewidth}
	\begin{tcolorbox}[colframe=yellow,colback=yellow!15]
		\begin{center}
			\begin{tabular}{c}
				{\Huge \texttt{postit} [fr]}\\
				\\
				{\LARGE Des petits Post-It,} \\
				\\
				{\LARGE avec \textsf{tcolorbox} ou \textsf{Ti\textit{k}Z}.} \\
			\end{tabular}
			
			\bigskip
			
			{\small \texttt{Version \TPversion{} -- \TPdate}}
		\end{center}
	\end{tcolorbox}
\end{minipage}
\end{center}

\begin{center}
	\begin{tabular}{c}
	\texttt{Cédric Pierquet}\\
	{\ttfamily c pierquet -- at -- outlook . fr}\\
	\texttt{\url{https://github.com/cpierquet/postit}}
\end{tabular}
\end{center}

\vspace{0.25cm}

{$\blacktriangleright$~~Placer/personnaliser/nommer des Post-It ou des \textit{mini-}Post-It.}

\vspace{0.25cm}

{$\blacktriangleright$~~Gestion de la taille, de l'inclinaison, de la décoration.}

\vspace{1cm}

\begin{PostIt}[RappelPostIt=PI1]<center>
	Ceci est un petit Post-It ! Pour rappeler par exemple que \[(a+b)^2=a^2+2ab+b^2.\]
\end{PostIt}

\begin{PostIt}[Rendu=tikz,Largeur=8cm,Couleur=orange,Attache=Trombone,CouleurAttache=blue,Inclinaison=-5,AlignementPostIt=center,Titre={- Titre -},PoliceTitre={\color{blue!50!black}\bfseries\small\sffamily},RappelPostIt=PI2]
\lipsum[1][1-4]

\end{PostIt}
\hfill
\begin{PostIt}[Hauteur=6cm,AlignementV=center,Couleur=pink,CouleurAttache=blue,Inclinaison=15,Coin,AlignementPostIt=center,Attache=Scotch,RappelPostIt=PI3]
\lipsum[1][1-4]
\end{PostIt}

\begin{tikzpicture}[remember picture,overlay]
	\draw[very thick,->,>=latex] (PI1-S)to[out=-90,in=90](PI2-N) ;
	\draw[very thick,lime,densely dashed,->,>=latex] (PI2-E)to[out=0,in=180](PI3-S-O) ;
\end{tikzpicture}

\vspace{0.5cm}

%\hfill{}\textit{Merci à Denis Bitouzé et à Gilles Le Bourhis pour leurs retours et idées !}

\smallskip

\vfill

\hrule

\medskip

\TableauDocumentation

\medskip

\hrule

\medskip

\newpage

\phantomsection
\hypertarget{matoc}{}

\tableofcontents

\vfill

\section{Historique}

\verb|v0.1.3|~:~~~~Les Post-It ont désormais un nom pour réutilisation ultérieure.

\verb|v0.1.2|~:~~~~Ajout des clés en langue anglaise.

\verb|v0.1.1|~:~~~~Ajout d'un \motcletex!\vphantom! pour les \textit{mini-}Post-It (hauteur \textit{uniforme}) + Correction de bugs.

\verb|      |~:~~~~Moteurs de rendu alternatif en \TikZ{} + Ajout d'un titre éventuel .

\verb|v0.1.0|~:~~~~Version initiale.

\newpage

\section{Le package postit}

\subsection{Introduction}

\begin{noteblock}
Le package propose de quoi afficher, dans son document \LaTeX, un Post-It (créé à l'aide de \packagetex!tcolorbox! ou de \packagetex!tikz!), avec la possibilité :

\begin{itemize}
	\item de spécifier les dimensions, la couleur ;
	\item de rajouter une \textit{attache} comme un trombone ou une punaise ;
	\item de personnaliser les bordure et le coin ;
	\item réutiliser des points d'ancrage du Post-It pour décorations ultérieures.
\end{itemize}

Le package propose également de quoi créer un \textit{mini-}Post-It (créé à l'aide d'une \motcletex!tcbox!), avec la possibilité de gérer la couleur et l'ombre.
\end{noteblock}

\subsection{Chargement du package, packages utilisés}

\begin{importantblock}
Le package se charge, de manière classique, dans le préambule.

Il n'existe pas d'option pour le package, et \packagetex!xcolor! n'est pas chargé.
\end{importantblock}

\begin{DemoCode}[listing only]
\documentclass{article}
\usepackage{postit}

\end{DemoCode}

\begin{noteblock}
\packagetex!postit! charge les packages suivantes :

\begin{itemize}
	\item \packagetex!tcolorbox! avec la librairie \packagetex!tcbox.skins! ;
	\item les librairies \packagetex!tikz! :
	\begin{itemize}
		\item \packagetex!tikz.calc! ;
		\item \packagetex!tikz.babel! ;
		\item \packagetex!tikz.decorations! ;
		\item \packagetex!tikz.decorations.pathmorphing! ;
	\end{itemize}
	\item \packagetex!settobox!, \packagetex!xstring!, \packagetex!varwidth! et \packagetex!simplekv!.
\end{itemize}

Il est compatible avec les compilations usuelles en \textsf{latex}, \textsf{pdflatex}, \textsf{lualatex} ou \textsf{xelatex}.
\end{noteblock}

\subsection{Compatibilité}

\begin{cautionblock}
Si un autre package charge \packagetex!tcolorbox!, et notamment avec l'option \Cle{[most]}, il vaut mieux charger \packagetex!postit! après, afin d'éviter un \motcletex!option clash error...!.
\end{cautionblock}

\begin{DemoCode}[listing only]
\documentclass{article}
\usepackage[<librairies>]{tcolorbox}
\usepackage{postit}
...

\end{DemoCode}

\vfill~

\pagebreak

\section{Environnement Post-It}

\subsection{Environnement et fonctionnement global}

\begin{cautionblock}
L'environnement dédié à la création du Post-It est \packagetex!PostIt!.

Il fonctionne avec un système de clés, entre \texttt{[...]} et il est possible, entre \texttt{<...>} de spécifier des options à la \motcletex!tcbox!, en langage \textsf{tcbox} (inutile avec le rendu en \motcletex!tikz!) !
\end{cautionblock}

\begin{DemoCode}[listing only]
\begin{PostIt}[clés]<options tcbox>
...
...
\end{PostIt}
\end{DemoCode}

\begin{noteblock}
Comme indiqué dans l'introduction, le Post-It est créé à l'aide d'un environnement \motcletex!tcbox! ou d'un environnement \motcletex!tikz!.

La majorité des (multiples) paramètres d'une \motcletex!tcbox! et d'une figuure \motcletex!tikz! sont fixés par le code, mais il est possible de spécifier certaines caractéristiques esthétiques du Post-It !
\end{noteblock}

\begin{DemoCode}[]
%sortie par défaut (rendu tcbox), avec un paragraphe issu du package lipsum
\begin{PostIt}
\lipsum[1][1-2]
\end{PostIt}
\end{DemoCode}

\begin{DemoCode}[]
%sortie rendu tikz, avec un paragraphe issu du package lipsum
\begin{PostIt}[Rendu=tikz]
	\lipsum[1][1-2]
\end{PostIt}
%sortie rendu tikzv2, avec un paragraphe issu du package lipsum
\begin{PostIt}[Rendu=tikzv2]
	\lipsum[1][1-2]
\end{PostIt}
\end{DemoCode}

\begin{tipblock}
Les éventuelles couleurs choisies devront être données de manière \textit{unique}, sans utiliser les \textit{mélanges} (avec \motcletex|CouleurA!...!CouleurB|) que propose le package \packagetex!xcolor!. 

Toutefois, toute couleur précédemment définie pourra être utilisée pour le Post-It.
\end{tipblock}

\begin{tipblock}
Le Post-It créé pourra être intégré dans une \motcletex!minipage! ou un \motcletex!wrapstuff! si besoin.

Pour l'alignement horizontal, il est conseillé d'utiliser des commandes dédiées comme \motcletex!\hfill! ou des environnements dédiés comme \motcletex!flush...!.
\end{tipblock}

\begin{warningblock}
Avec une \textit{attache} qui "déborde" verticalement (rendu \textsf{tcbox}), il sera sans doute nécessaire d'ajuster l'espacement vertical précédant le Post-It pour éviter un éventuel chevauchement.
\end{warningblock}

\subsection{Clés et options}

\begin{tipblock}
Le premier argument, optionnel et entre \texttt{[...]}, propose les \Cle{clés} suivantes :

\begin{itemize}
	\item \cmaj{0.1.3} \Cle{RappelPostIt} : nom (pour du code \TikZ{} ultérieur) du Post-It ; \hfill{}défaut : \Cle{PostIt}
	\item \Cle{Largeur} : largeur (en cm) du Post-It ; \hfill{}défaut : \Cle{6cm}
	\item \Cle{Couleur} : couleur du Post-It (la bordure sera plus foncée) ; \hfill{}défaut : \Cle{yellow}
	\item \Cle{Hauteur} : hauteur (en cm si déclarée) du Post-It (par défaut elle est \textit{automatique}) ;
	
	\hfill{}défaut : \Cle{auto}
	\item \cmaj{0.1.1} \Cle{Rendu} : moteur du rendu parmi \Cle{tcbox / tikz / tikv2} ; \hfill{}défaut : \Cle{tcbox}
	\item \Cle{Inclinaison} : inclinaison du Post-It ; \hfill{}défaut : \Cle{0}
	\item \Cle{Ombre} : booléen pour afficher une ombre portée ; \hfill{}défaut : \Cle{true}
	\item \Cle{Bordure} : booléen pour afficher une fine bordure ; \hfill{}défaut : \Cle{true}
	\item \Cle{Coin} : booléen pour afficher un coin corné ; \hfill{}défaut : \Cle{false}
	\item \cmaj{0.1.1} \Cle{Attache} : choix de la décoration, parmi \Cle{Trombone / Punaise / Non / Scotch} ;
	
	\hfill{}défaut : \Cle{Punaise}
	\item \Cle{CouleurAttache} : couleur de l'attache ; \hfill{}défaut : \Cle{red}
	\item \Cle{DecalAttache} : décalage horizontal (sans unité, mais en cm) de l'attache par rapport à sa position initiale (au centre pour la punaise, à 1~cm du bord droit pour le trombone) ;
	
	\hfill{}défaut : \Cle{0}
	\item \cmaj{0.1.1} \Cle{Titre} : rajouter un titre (en 1ère ligne et/ou sous l'Attache) ; \hfill{}défaut : \Cle{vide}
	\item \cmaj{0.1.1} \Cle{PoliceTitre} : police du titre ; \hfill{}défaut : \Cle{\textbackslash normalfont\textbackslash normalfont}
	\item \cmaj{0.1.1} \Cle{ExtraMargeDroite} : rajoute (en rendu \packagetex!tikz!, et en cm) une marge à droite ;
	
	\hfill{}défaut : \Cle{0cm}
	\item \Cle{AlignementV} : gère l'alignement vertical dans le Post-It (parmi \Cle{top/center/bottom}) ;
	
	\hfill{}défaut : \Cle{top}
	\item \Cle{AlignementH} : gère l'alignement horizontal dans le Post-It (parmi \Cle{left/center/right/justify}) ;
	
	\hfill{}défaut : \Cle{justify}
	\item \Cle{AlignementPostIt} : gère l'alignement vertical global du Post-It (parmi \Cle{top/center/bottom}).
	
	\hfill{}défaut : \Cle{bottom}
\end{itemize}
\vspace*{-\baselineskip}\leavevmode
\end{tipblock}

\begin{tipblock}
Le second argument, optionnel et entre \texttt{<...>} correspond à des options spécifiques à passer à la \motcletex!tcolorbox!, en langage \textsf{tcbox} (inutile si le rendu est \motcletex!tikz!).

Elles permettent de modifier localement des options non paramétrées par des clés présentées précédemment.
\end{tipblock}

\subsection{Fonctionnement des points d'ancrage}

\begin{tipblock}
En plus du Post-It, le package \packagetex!postit! crée huit points d'ancrage pour le Post-It, qui seront nommés :

\begin{itemize}
	\item \motcletex!(<nom>-N)!, \motcletex!(<nom>-E)!, \motcletex!(<nom>-S)! et \motcletex!(<nom>-O)! pour les points Nord/Est/Sud/Ouest ;
	\item \motcletex!(<nom>-N-O)!, \motcletex!(<nom>-N-E)!, \motcletex!(<nom>-S-E)! et \motcletex!(<nom>-S-O)! pour les points Nord Est/Nord Ouest/\ldots.
\end{itemize}
\end{tipblock}

\begin{DemoCode}[]
\begin{center}
\begin{PostIt}[Inclinaison=10,Attache=Non,Rendu=tikz,RappelPostIt=MaPetiteNote1]
	\lipsum[1][1-2]
\end{PostIt}
\end{center}
\end{DemoCode}

\begin{tikzpicture}[remember picture,overlay]
	\foreach \dir/\pos in {N-O/above left,N/above,N-E/above right,E/right, S-E/below right,S/below,S-O/below left,O/left} 
	{%
		\draw[draw=blue,fill=red] (MaPetiteNote1-\dir) circle[radius=2pt] node[text=gray,\pos,font=\scriptsize\ttfamily] {MaPetiteNote1-\dir};%
	}
\end{tikzpicture}

\begin{DemoCode}[]
\begin{PostIt}[RappelPostIt=NoteY]<center>
	Ceci est un petit Post-It ! Pour rappeler par exemple que \[(a+b)^2=a^2+2ab+b^2.\]
\end{PostIt}\\
\begin{PostIt}[Rendu=tikz,Largeur=8cm,Couleur=blue,Inclinaison=-5,RappelPostIt=NoteZ]
	\lipsum[1][1-2]
\end{PostIt}

\begin{tikzpicture}[remember picture,overlay]
	\draw[very thick,->,>=latex] (NoteY-S)to[out=-90,in=90](NoteZ-N) ;
\end{tikzpicture}
\end{DemoCode}

\subsection{Exemples}

\begin{DemoCode}[]
\begin{PostIt}%moteur de rendu tcbox (défaut)
	[Couleur=cyan,Attache=Trombone,Largeur=10cm,Inclinaison=10]<center,right=1.5cm>
\lipsum[1][1-3]
\end{PostIt}
\end{DemoCode}

\begin{DemoCode}[]
\hfill\begin{PostIt}%moteur de rendu tikz
	[Rendu=tikz,Couleur=violet,Largeur=9cm,Inclinaison=-10,Attache=Trombone,
	CouleurAttache=black,ExtraMargeDroite=1cm,Titre={Petit Titre},
	PoliceTitre={\color{white}\bfseries\small\sffamily}]
\lipsum[1][1-3]
\end{PostIt}\hfill~
\end{DemoCode}

\begin{DemoCode}[]
\hfill\begin{PostIt}%moteur de rendu tikzv2
	[Rendu=tikzv2,Couleur=orange,Largeur=9cm,Inclinaison=-10,Attache=Scotch, 	Titre={Essai},
	PoliceTitre={\color{blue!50!black}\bfseries\itshape\small\ttfamily}]
\lipsum[1][1-3]
\end{PostIt}\hfill~
\end{DemoCode}

\begin{DemoCode}[]
%usepackage{wrapstuff}
\begin{wrapstuff}[r,top=1]
\begin{PostIt}[Inclinaison=5,Coin,Couleur=pink,CouleurAttache=blue,Bordure=false]
\lipsum[1][1-2]
\end{PostIt}
\end{wrapstuff}

\lipsum[1]
\end{DemoCode}

\begin{DemoCode}[]
%usepackage{wrapstuff}
\begin{wrapstuff}[r,top=1]
\begin{PostIt}[Inclinaison=5,Rendu=tikz,Couleur=pink, CouleurAttache=blue,Bordure=false]
\lipsum[1][1-2]
\end{PostIt}
\end{wrapstuff}

\lipsum[1]
\end{DemoCode}

\begin{DemoCode}[]
%usepackage{wrapstuff}
\begin{wrapstuff}[r,top=1]
\begin{PostIt}[Inclinaison=5,Rendu=tikzv2,Attache=Scotch,Couleur=pink]
\lipsum[1][1-2]
\end{PostIt}
\end{wrapstuff}

\lipsum[1]
\end{DemoCode}

\begin{DemoCode}[]
Un petit Post-It aligné à droite, et centré verticalement :
%
\hfill\begin{PostIt}[Inclinaison=-10,Couleur=orange,Largeur=5cm,Hauteur=5cm, AlignementV=center,Coin,CouleurAttache=yellow, DecalAttache=-1,AlignementPostIt=center]

\textsf{\small\lipsum[1][1-2]}
\[\mathsf{\displaystyle\sum_{k=1}^{n} k = \dfrac{n(n+1)}{2}}\]
\end{PostIt}
\end{DemoCode}

%\begin{DemoCode}[]
%Un petit Post-It aligné à droite, et centré verticalement :
%%
%\hfill\begin{PostIt}[Inclinaison=-10,Couleur=orange,Largeur=5cm,Hauteur=5cm, AlignementV=center,Rendu=tikz,Attache=Non,AlignementPostIt=center]
%
%\textsf{\small\lipsum[1][1-2]}
%\[\mathsf{\displaystyle\sum_{k=1}^{n} k = \dfrac{n(n+1)}{2}}\]
%\end{PostIt}
%\end{DemoCode}
%
%\vfill~

\pagebreak

\section{Post-It simple, en ligne}

\subsection{Commande et fonctionnement global}

\begin{cautionblock}
La commande dédiée à la création du \textit{mini-}Post-It est \motcletex!MiniPostIt!.

Elle fonctionne sous forme autonome, avec uniquement la couleur en \Cle{option}.

\smallskip

Cette fois-ci le \textit{mini-} Post-It est créé à l'aide d'une commande \motcletex!tcbox!.

\smallskip

Les dimensions ne sont pas modifiables, et un \motcletex!\vphantom! est inséré au début de la \motcletex!tcbox! afin d'harmoniser la hauteur.
\end{cautionblock}

\begin{DemoCode}[listing only]
\MiniPostIt(*)[couleur]{contenu}
\end{DemoCode}

\subsection{Arguments}

\begin{noteblock}
La version étoilée active l'ombre du \textit{mini-}Post-It.

La couleur (\Cle{yellow}), est gérée par l'argument optionnel entre \texttt{[...]}.
\end{noteblock}

\subsection{Exemples}

\begin{DemoCode}[]
On va travailler sur une équation diophantienne du type $ax+by=c$.

On va utiliser le \MiniPostIt*[orange]{théorème de Bezout}, le \MiniPostIt{théorème de Gauss} sans oublier la \MiniPostIt*[cyan]{réciproque}.

Le schéma de résolution est classique, et assez simple à appréhender !
\end{DemoCode}

\pagebreak

\section{Résumé des styles}

\subsection{Moteur de rendu tcbox}

\begin{DemoCode}[text only]
\hfill\begin{PostIt}
\texttt{Ombre/Épingle/Bordure}
\end{PostIt}
\begin{PostIt}[Ombre=false]
\texttt{Épingle/Bordure}
\end{PostIt}\hfill~

\medskip

\hfill\begin{PostIt}[Bordure=false]
\texttt{Ombre/Épingle}
\end{PostIt}
\begin{PostIt}[Bordure=false,Ombre=false]
\texttt{Épingle}
\end{PostIt}\hfill~

\medskip

\hfill\begin{PostIt}[Attache=Trombone]
\texttt{Ombre/Trombone/Bordure}\\
~
\end{PostIt}
\begin{PostIt}[Attache=Scotch]
\texttt{Ombre/Scotch/Bordure}\\
~
\end{PostIt}\hfill~

\medskip

\hfill\begin{PostIt}[Attache=Non]
\texttt{Ombre/Bordure}
\end{PostIt}
\begin{PostIt}[Coin,Attache=Non]
\texttt{Ombre/Bordure/Coin}
\end{PostIt}\hfill~

\vspace{1cm}

\hfill\begin{PostIt}[Titre={Lipsum[1][1-4]},PoliceTitre={\large\sffamily},Inclinaison=5,Couleur=pink,Hauteur=6cm,Attache=Scotch,AlignementV=center,Coin]
\lipsum[1][1-4]
\end{PostIt}\hfill~
\end{DemoCode}

\pagebreak

\subsection{Moteur de rendu tikz}

\begin{DemoCode}[text only]
\hfill\begin{PostIt}[Rendu=tikz]
\texttt{Ombre/Épingle/Bordure}
\end{PostIt}
\begin{PostIt}[Ombre=false,Rendu=tikz]
\texttt{Épingle/Bordure}
\end{PostIt}\hfill~

\medskip

\hfill\begin{PostIt}[Bordure=false,Rendu=tikz]
\texttt{Ombre/Épingle}
\end{PostIt}
\begin{PostIt}[Bordure=false,Ombre=false,Rendu=tikz]
\texttt{Épingle}
\end{PostIt}\hfill~

\medskip

\hfill\begin{PostIt}[Attache=Trombone,Rendu=tikz]
\texttt{Ombre/Trombone/Bordure}\\
~
\end{PostIt}
\begin{PostIt}[Attache=Scotch,Rendu=tikz]
\texttt{Ombre/Scotch/Bordure}\\
~
\end{PostIt}\hfill~

\medskip

\hfill\begin{PostIt}[Attache=Non,Rendu=tikz]
\texttt{Ombre/Bordure}
\end{PostIt}\hfill~

\vspace{1cm}

\hfill\begin{PostIt}[Rendu=tikz,Titre={Lipsum[1][1-4]},PoliceTitre={\large\sffamily},Inclinaison=5,Couleur=pink,Hauteur=6cm,Attache=Scotch,AlignementV=center,Coin]
\lipsum[1][1-4]
\end{PostIt}\hfill~
\end{DemoCode}

\subsection{Moteur de rendu tikzv2}

\begin{DemoCode}[text only]
\hfill\begin{PostIt}[Rendu=tikzv2]
\texttt{Ombre/Épingle/Bordure}
\end{PostIt}
\begin{PostIt}[Ombre=false,Rendu=tikzv2]
\texttt{Épingle/Bordure}
\end{PostIt}\hfill~

\medskip

\hfill\begin{PostIt}[Bordure=false,Rendu=tikzv2]
\texttt{Ombre/Épingle}
\end{PostIt}
\begin{PostIt}[Bordure=false,Ombre=false,Rendu=tikzv2]
\texttt{Épingle}
\end{PostIt}\hfill~

\medskip

\hfill\begin{PostIt}[Attache=Trombone,Rendu=tikzv2]
\texttt{Ombre/Trombone/Bordure}\\
~
\end{PostIt}
\begin{PostIt}[Attache=Scotch,Rendu=tikzv2]
\texttt{Ombre/Scotch/Bordure}\\
~
\end{PostIt}\hfill~

\medskip

\hfill\begin{PostIt}[Attache=Non,Rendu=tikzv2]
\texttt{Ombre/Bordure}
\end{PostIt}\hfill~

\vspace{1cm}

\hfill\begin{PostIt}[Rendu=tikzv2,Titre={Lipsum[1][1-4]},PoliceTitre={\large\sffamily},Inclinaison=5,Couleur=pink,Hauteur=6cm,Attache=Scotch,AlignementV=center,Coin]
\lipsum[1][1-4]
\end{PostIt}\hfill~
\end{DemoCode}



\end{document}